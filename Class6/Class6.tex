% Options for packages loaded elsewhere
\PassOptionsToPackage{unicode}{hyperref}
\PassOptionsToPackage{hyphens}{url}
\PassOptionsToPackage{dvipsnames,svgnames,x11names}{xcolor}
%
\documentclass[
  letterpaper,
  DIV=11,
  numbers=noendperiod]{scrartcl}

\usepackage{amsmath,amssymb}
\usepackage{lmodern}
\usepackage{iftex}
\ifPDFTeX
  \usepackage[T1]{fontenc}
  \usepackage[utf8]{inputenc}
  \usepackage{textcomp} % provide euro and other symbols
\else % if luatex or xetex
  \usepackage{unicode-math}
  \defaultfontfeatures{Scale=MatchLowercase}
  \defaultfontfeatures[\rmfamily]{Ligatures=TeX,Scale=1}
\fi
% Use upquote if available, for straight quotes in verbatim environments
\IfFileExists{upquote.sty}{\usepackage{upquote}}{}
\IfFileExists{microtype.sty}{% use microtype if available
  \usepackage[]{microtype}
  \UseMicrotypeSet[protrusion]{basicmath} % disable protrusion for tt fonts
}{}
\makeatletter
\@ifundefined{KOMAClassName}{% if non-KOMA class
  \IfFileExists{parskip.sty}{%
    \usepackage{parskip}
  }{% else
    \setlength{\parindent}{0pt}
    \setlength{\parskip}{6pt plus 2pt minus 1pt}}
}{% if KOMA class
  \KOMAoptions{parskip=half}}
\makeatother
\usepackage{xcolor}
\setlength{\emergencystretch}{3em} % prevent overfull lines
\setcounter{secnumdepth}{-\maxdimen} % remove section numbering
% Make \paragraph and \subparagraph free-standing
\ifx\paragraph\undefined\else
  \let\oldparagraph\paragraph
  \renewcommand{\paragraph}[1]{\oldparagraph{#1}\mbox{}}
\fi
\ifx\subparagraph\undefined\else
  \let\oldsubparagraph\subparagraph
  \renewcommand{\subparagraph}[1]{\oldsubparagraph{#1}\mbox{}}
\fi

\usepackage{color}
\usepackage{fancyvrb}
\newcommand{\VerbBar}{|}
\newcommand{\VERB}{\Verb[commandchars=\\\{\}]}
\DefineVerbatimEnvironment{Highlighting}{Verbatim}{commandchars=\\\{\}}
% Add ',fontsize=\small' for more characters per line
\usepackage{framed}
\definecolor{shadecolor}{RGB}{241,243,245}
\newenvironment{Shaded}{\begin{snugshade}}{\end{snugshade}}
\newcommand{\AlertTok}[1]{\textcolor[rgb]{0.68,0.00,0.00}{#1}}
\newcommand{\AnnotationTok}[1]{\textcolor[rgb]{0.37,0.37,0.37}{#1}}
\newcommand{\AttributeTok}[1]{\textcolor[rgb]{0.40,0.45,0.13}{#1}}
\newcommand{\BaseNTok}[1]{\textcolor[rgb]{0.68,0.00,0.00}{#1}}
\newcommand{\BuiltInTok}[1]{\textcolor[rgb]{0.00,0.23,0.31}{#1}}
\newcommand{\CharTok}[1]{\textcolor[rgb]{0.13,0.47,0.30}{#1}}
\newcommand{\CommentTok}[1]{\textcolor[rgb]{0.37,0.37,0.37}{#1}}
\newcommand{\CommentVarTok}[1]{\textcolor[rgb]{0.37,0.37,0.37}{\textit{#1}}}
\newcommand{\ConstantTok}[1]{\textcolor[rgb]{0.56,0.35,0.01}{#1}}
\newcommand{\ControlFlowTok}[1]{\textcolor[rgb]{0.00,0.23,0.31}{#1}}
\newcommand{\DataTypeTok}[1]{\textcolor[rgb]{0.68,0.00,0.00}{#1}}
\newcommand{\DecValTok}[1]{\textcolor[rgb]{0.68,0.00,0.00}{#1}}
\newcommand{\DocumentationTok}[1]{\textcolor[rgb]{0.37,0.37,0.37}{\textit{#1}}}
\newcommand{\ErrorTok}[1]{\textcolor[rgb]{0.68,0.00,0.00}{#1}}
\newcommand{\ExtensionTok}[1]{\textcolor[rgb]{0.00,0.23,0.31}{#1}}
\newcommand{\FloatTok}[1]{\textcolor[rgb]{0.68,0.00,0.00}{#1}}
\newcommand{\FunctionTok}[1]{\textcolor[rgb]{0.28,0.35,0.67}{#1}}
\newcommand{\ImportTok}[1]{\textcolor[rgb]{0.00,0.46,0.62}{#1}}
\newcommand{\InformationTok}[1]{\textcolor[rgb]{0.37,0.37,0.37}{#1}}
\newcommand{\KeywordTok}[1]{\textcolor[rgb]{0.00,0.23,0.31}{#1}}
\newcommand{\NormalTok}[1]{\textcolor[rgb]{0.00,0.23,0.31}{#1}}
\newcommand{\OperatorTok}[1]{\textcolor[rgb]{0.37,0.37,0.37}{#1}}
\newcommand{\OtherTok}[1]{\textcolor[rgb]{0.00,0.23,0.31}{#1}}
\newcommand{\PreprocessorTok}[1]{\textcolor[rgb]{0.68,0.00,0.00}{#1}}
\newcommand{\RegionMarkerTok}[1]{\textcolor[rgb]{0.00,0.23,0.31}{#1}}
\newcommand{\SpecialCharTok}[1]{\textcolor[rgb]{0.37,0.37,0.37}{#1}}
\newcommand{\SpecialStringTok}[1]{\textcolor[rgb]{0.13,0.47,0.30}{#1}}
\newcommand{\StringTok}[1]{\textcolor[rgb]{0.13,0.47,0.30}{#1}}
\newcommand{\VariableTok}[1]{\textcolor[rgb]{0.07,0.07,0.07}{#1}}
\newcommand{\VerbatimStringTok}[1]{\textcolor[rgb]{0.13,0.47,0.30}{#1}}
\newcommand{\WarningTok}[1]{\textcolor[rgb]{0.37,0.37,0.37}{\textit{#1}}}

\providecommand{\tightlist}{%
  \setlength{\itemsep}{0pt}\setlength{\parskip}{0pt}}\usepackage{longtable,booktabs,array}
\usepackage{calc} % for calculating minipage widths
% Correct order of tables after \paragraph or \subparagraph
\usepackage{etoolbox}
\makeatletter
\patchcmd\longtable{\par}{\if@noskipsec\mbox{}\fi\par}{}{}
\makeatother
% Allow footnotes in longtable head/foot
\IfFileExists{footnotehyper.sty}{\usepackage{footnotehyper}}{\usepackage{footnote}}
\makesavenoteenv{longtable}
\usepackage{graphicx}
\makeatletter
\def\maxwidth{\ifdim\Gin@nat@width>\linewidth\linewidth\else\Gin@nat@width\fi}
\def\maxheight{\ifdim\Gin@nat@height>\textheight\textheight\else\Gin@nat@height\fi}
\makeatother
% Scale images if necessary, so that they will not overflow the page
% margins by default, and it is still possible to overwrite the defaults
% using explicit options in \includegraphics[width, height, ...]{}
\setkeys{Gin}{width=\maxwidth,height=\maxheight,keepaspectratio}
% Set default figure placement to htbp
\makeatletter
\def\fps@figure{htbp}
\makeatother

\KOMAoption{captions}{tableheading}
\makeatletter
\makeatother
\makeatletter
\makeatother
\makeatletter
\@ifpackageloaded{caption}{}{\usepackage{caption}}
\AtBeginDocument{%
\ifdefined\contentsname
  \renewcommand*\contentsname{Table of contents}
\else
  \newcommand\contentsname{Table of contents}
\fi
\ifdefined\listfigurename
  \renewcommand*\listfigurename{List of Figures}
\else
  \newcommand\listfigurename{List of Figures}
\fi
\ifdefined\listtablename
  \renewcommand*\listtablename{List of Tables}
\else
  \newcommand\listtablename{List of Tables}
\fi
\ifdefined\figurename
  \renewcommand*\figurename{Figure}
\else
  \newcommand\figurename{Figure}
\fi
\ifdefined\tablename
  \renewcommand*\tablename{Table}
\else
  \newcommand\tablename{Table}
\fi
}
\@ifpackageloaded{float}{}{\usepackage{float}}
\floatstyle{ruled}
\@ifundefined{c@chapter}{\newfloat{codelisting}{h}{lop}}{\newfloat{codelisting}{h}{lop}[chapter]}
\floatname{codelisting}{Listing}
\newcommand*\listoflistings{\listof{codelisting}{List of Listings}}
\makeatother
\makeatletter
\@ifpackageloaded{caption}{}{\usepackage{caption}}
\@ifpackageloaded{subcaption}{}{\usepackage{subcaption}}
\makeatother
\makeatletter
\@ifpackageloaded{tcolorbox}{}{\usepackage[many]{tcolorbox}}
\makeatother
\makeatletter
\@ifundefined{shadecolor}{\definecolor{shadecolor}{rgb}{.97, .97, .97}}
\makeatother
\makeatletter
\makeatother
\ifLuaTeX
  \usepackage{selnolig}  % disable illegal ligatures
\fi
\IfFileExists{bookmark.sty}{\usepackage{bookmark}}{\usepackage{hyperref}}
\IfFileExists{xurl.sty}{\usepackage{xurl}}{} % add URL line breaks if available
\urlstyle{same} % disable monospaced font for URLs
\hypersetup{
  pdftitle={Class6},
  pdfauthor={Uwaysah},
  colorlinks=true,
  linkcolor={blue},
  filecolor={Maroon},
  citecolor={Blue},
  urlcolor={Blue},
  pdfcreator={LaTeX via pandoc}}

\title{Class6}
\author{Uwaysah}
\date{}

\begin{document}
\maketitle
\ifdefined\Shaded\renewenvironment{Shaded}{\begin{tcolorbox}[borderline west={3pt}{0pt}{shadecolor}, interior hidden, sharp corners, enhanced, boxrule=0pt, frame hidden, breakable]}{\end{tcolorbox}}\fi

\begin{quote}
To open: go into BIMM 143 folder, go to file, new project, new
directory, new project, name it, new file, quarto document, title and
author, click create empty document.
\end{quote}

\begin{quote}
To open a line of code in quarto, have to click +C button
\end{quote}

\begin{Shaded}
\begin{Highlighting}[]
\NormalTok{student1 }\OtherTok{\textless{}{-}} \FunctionTok{c}\NormalTok{(}\DecValTok{100}\NormalTok{, }\DecValTok{100}\NormalTok{, }\DecValTok{100}\NormalTok{, }\DecValTok{100}\NormalTok{, }\DecValTok{100}\NormalTok{, }\DecValTok{100}\NormalTok{, }\DecValTok{100}\NormalTok{, }\DecValTok{90}\NormalTok{)}
\NormalTok{student2 }\OtherTok{\textless{}{-}} \FunctionTok{c}\NormalTok{(}\DecValTok{100}\NormalTok{, }\ConstantTok{NA}\NormalTok{, }\DecValTok{90}\NormalTok{, }\DecValTok{90}\NormalTok{, }\DecValTok{90}\NormalTok{, }\DecValTok{90}\NormalTok{, }\DecValTok{97}\NormalTok{, }\DecValTok{80}\NormalTok{)}
\NormalTok{student3 }\OtherTok{\textless{}{-}} \FunctionTok{c}\NormalTok{(}\DecValTok{90}\NormalTok{, }\ConstantTok{NA}\NormalTok{, }\ConstantTok{NA}\NormalTok{, }\ConstantTok{NA}\NormalTok{, }\ConstantTok{NA}\NormalTok{, }\ConstantTok{NA}\NormalTok{, }\ConstantTok{NA}\NormalTok{, }\ConstantTok{NA}\NormalTok{)}

\FunctionTok{mean}\NormalTok{(student1)}
\end{Highlighting}
\end{Shaded}

\begin{verbatim}
[1] 98.75
\end{verbatim}

\begin{Shaded}
\begin{Highlighting}[]
\FunctionTok{mean}\NormalTok{(student2)}
\end{Highlighting}
\end{Shaded}

\begin{verbatim}
[1] NA
\end{verbatim}

\begin{quote}
every function has a name, an argument (inputs), and a body
\end{quote}

\begin{Shaded}
\begin{Highlighting}[]
\FunctionTok{mean}\NormalTok{(student2, }\AttributeTok{na.rm=}\ConstantTok{TRUE}\NormalTok{)}
\end{Highlighting}
\end{Shaded}

\begin{verbatim}
[1] 91
\end{verbatim}

\begin{Shaded}
\begin{Highlighting}[]
\FunctionTok{mean}\NormalTok{(student3, }\AttributeTok{na.rm=}\ConstantTok{TRUE}\NormalTok{)}
\end{Highlighting}
\end{Shaded}

\begin{verbatim}
[1] 90
\end{verbatim}

\begin{Shaded}
\begin{Highlighting}[]
\NormalTok{student2}
\end{Highlighting}
\end{Shaded}

\begin{verbatim}
[1] 100  NA  90  90  90  90  97  80
\end{verbatim}

\begin{Shaded}
\begin{Highlighting}[]
\FunctionTok{is.na}\NormalTok{(student2)}
\end{Highlighting}
\end{Shaded}

\begin{verbatim}
[1] FALSE  TRUE FALSE FALSE FALSE FALSE FALSE FALSE
\end{verbatim}

\begin{Shaded}
\begin{Highlighting}[]
\NormalTok{student2[}\FunctionTok{c}\NormalTok{(T,F,T,T,T,T,T,T)]}
\end{Highlighting}
\end{Shaded}

\begin{verbatim}
[1] 100  90  90  90  90  97  80
\end{verbatim}

\begin{Shaded}
\begin{Highlighting}[]
\NormalTok{student2[}\FunctionTok{is.na}\NormalTok{(student2)]}
\end{Highlighting}
\end{Shaded}

\begin{verbatim}
[1] NA
\end{verbatim}

\begin{Shaded}
\begin{Highlighting}[]
\FunctionTok{which}\NormalTok{(}\FunctionTok{is.na}\NormalTok{(student2))}
\end{Highlighting}
\end{Shaded}

\begin{verbatim}
[1] 2
\end{verbatim}

\begin{Shaded}
\begin{Highlighting}[]
\NormalTok{student2[}\FunctionTok{is.na}\NormalTok{(student2)] }\OtherTok{\textless{}{-}} \DecValTok{0}
\NormalTok{student2}
\end{Highlighting}
\end{Shaded}

\begin{verbatim}
[1] 100   0  90  90  90  90  97  80
\end{verbatim}

\begin{Shaded}
\begin{Highlighting}[]
\FunctionTok{is.na}\NormalTok{(student3)}
\end{Highlighting}
\end{Shaded}

\begin{verbatim}
[1] FALSE  TRUE  TRUE  TRUE  TRUE  TRUE  TRUE  TRUE
\end{verbatim}

\begin{Shaded}
\begin{Highlighting}[]
\NormalTok{student3[}\FunctionTok{is.na}\NormalTok{(student3)] }\OtherTok{\textless{}{-}}\DecValTok{0}
\NormalTok{student3}
\end{Highlighting}
\end{Shaded}

\begin{verbatim}
[1] 90  0  0  0  0  0  0  0
\end{verbatim}

\begin{Shaded}
\begin{Highlighting}[]
\NormalTok{x }\OtherTok{\textless{}{-}}\NormalTok{ student1}
\NormalTok{x[}\FunctionTok{is.na}\NormalTok{(x)] }\OtherTok{\textless{}{-}}\DecValTok{0}
\FunctionTok{mean}\NormalTok{(x)}
\end{Highlighting}
\end{Shaded}

\begin{verbatim}
[1] 98.75
\end{verbatim}

\begin{Shaded}
\begin{Highlighting}[]
\NormalTok{y }\OtherTok{\textless{}{-}}\NormalTok{ student2}
\NormalTok{y[}\FunctionTok{is.na}\NormalTok{(y)] }\OtherTok{\textless{}{-}}\DecValTok{0}
\FunctionTok{mean}\NormalTok{(y)}
\end{Highlighting}
\end{Shaded}

\begin{verbatim}
[1] 79.625
\end{verbatim}

\begin{Shaded}
\begin{Highlighting}[]
\NormalTok{z }\OtherTok{\textless{}{-}}\NormalTok{ student3}
\NormalTok{z[}\FunctionTok{is.na}\NormalTok{(z)] }\OtherTok{\textless{}{-}}\DecValTok{0}
\FunctionTok{mean}\NormalTok{(z)}
\end{Highlighting}
\end{Shaded}

\begin{verbatim}
[1] 11.25
\end{verbatim}

\begin{Shaded}
\begin{Highlighting}[]
\CommentTok{\#we want to drop the lowest score before taking the average grade.}
\NormalTok{x }\OtherTok{\textless{}{-}}\NormalTok{student3}
\NormalTok{x[}\SpecialCharTok{{-}}\FunctionTok{which.min}\NormalTok{(x)]}
\end{Highlighting}
\end{Shaded}

\begin{verbatim}
[1] 90  0  0  0  0  0  0
\end{verbatim}

\begin{Shaded}
\begin{Highlighting}[]
\FunctionTok{mean}\NormalTok{(x[}\SpecialCharTok{{-}}\FunctionTok{which.min}\NormalTok{(x)])}
\end{Highlighting}
\end{Shaded}

\begin{verbatim}
[1] 12.85714
\end{verbatim}

\begin{Shaded}
\begin{Highlighting}[]
\CommentTok{\#name in our case is "grade", input arguments are "student1" etc., and body is working snippet}
\NormalTok{grade }\OtherTok{\textless{}{-}} \ControlFlowTok{function}\NormalTok{(x) \{}
\NormalTok{  x[}\FunctionTok{is.na}\NormalTok{(x)] }\OtherTok{\textless{}{-}}\DecValTok{0}
\FunctionTok{mean}\NormalTok{(x[}\SpecialCharTok{{-}}\FunctionTok{which.min}\NormalTok{(x)])}
\NormalTok{\}}
\end{Highlighting}
\end{Shaded}

\begin{quote}
can I use this function now? You have to read the code first in R
\end{quote}

\begin{Shaded}
\begin{Highlighting}[]
\FunctionTok{grade}\NormalTok{(student1)}
\end{Highlighting}
\end{Shaded}

\begin{verbatim}
[1] 100
\end{verbatim}

\begin{Shaded}
\begin{Highlighting}[]
\FunctionTok{grade}\NormalTok{(student2)}
\end{Highlighting}
\end{Shaded}

\begin{verbatim}
[1] 91
\end{verbatim}

\begin{Shaded}
\begin{Highlighting}[]
\FunctionTok{grade}\NormalTok{(student3)}
\end{Highlighting}
\end{Shaded}

\begin{verbatim}
[1] 12.85714
\end{verbatim}

\begin{Shaded}
\begin{Highlighting}[]
\CommentTok{\#read a gradebook from online:}
\NormalTok{hw }\OtherTok{\textless{}{-}} \FunctionTok{read.csv}\NormalTok{(}\StringTok{"https://tinyurl.com/gradeinput"}\NormalTok{, }\AttributeTok{row.names =} \DecValTok{1}\NormalTok{)}
\NormalTok{hw}
\end{Highlighting}
\end{Shaded}

\begin{verbatim}
           hw1 hw2 hw3 hw4 hw5
student-1  100  73 100  88  79
student-2   85  64  78  89  78
student-3   83  69  77 100  77
student-4   88  NA  73 100  76
student-5   88 100  75  86  79
student-6   89  78 100  89  77
student-7   89 100  74  87 100
student-8   89 100  76  86 100
student-9   86 100  77  88  77
student-10  89  72  79  NA  76
student-11  82  66  78  84 100
student-12 100  70  75  92 100
student-13  89 100  76 100  80
student-14  85 100  77  89  76
student-15  85  65  76  89  NA
student-16  92 100  74  89  77
student-17  88  63 100  86  78
student-18  91  NA 100  87 100
student-19  91  68  75  86  79
student-20  91  68  76  88  76
\end{verbatim}

\begin{Shaded}
\begin{Highlighting}[]
\CommentTok{\#Q1: Write a function grade() to determine an overall grade from a vector of student homework assignment scores dropping the lowest single score. If a student misses a homework (i.e. has an NA value) this can be used as a score to be potentially dropped. Your final function should be adquately explained with code comments and be able to work on an example class gradebook such as this one in CSV format: “https://tinyurl.com/gradeinput” [3pts]}

\CommentTok{\#apply(data, margin=1 where 1 means rows and 2 means columns, function)}

\NormalTok{ans }\OtherTok{\textless{}{-}} \FunctionTok{apply}\NormalTok{(hw, }\DecValTok{1}\NormalTok{, grade)}
\NormalTok{ans}
\end{Highlighting}
\end{Shaded}

\begin{verbatim}
 student-1  student-2  student-3  student-4  student-5  student-6  student-7 
     91.75      82.50      84.25      84.25      88.25      89.00      94.00 
 student-8  student-9 student-10 student-11 student-12 student-13 student-14 
     93.75      87.75      79.00      86.00      91.75      92.25      87.75 
student-15 student-16 student-17 student-18 student-19 student-20 
     78.75      89.50      88.00      94.50      82.75      82.75 
\end{verbatim}

\begin{quote}
note to self: to save a file while you're working on it, just click the
save button and name the file. It will be saved to RStudio, not your
desktop
\end{quote}

\begin{quote}
Q2: Using your grade() function and the supplied gradebook, Who is the
top scoring student overall in the gradebook? {[}3pts{]}
\end{quote}

\begin{Shaded}
\begin{Highlighting}[]
\NormalTok{ans[}\FunctionTok{which.max}\NormalTok{(ans)]}
\end{Highlighting}
\end{Shaded}

\begin{verbatim}
student-18 
      94.5 
\end{verbatim}

\begin{quote}
Q3: From your analysis of the gradebook, which homework was toughest on
students (i.e.~obtained the lowest scores overall? {[}2pts{]}
\end{quote}

\begin{Shaded}
\begin{Highlighting}[]
\NormalTok{ave.scores }\OtherTok{\textless{}{-}}\FunctionTok{apply}\NormalTok{(hw, }\DecValTok{2}\NormalTok{, mean, }\AttributeTok{na.rm=}\ConstantTok{TRUE}\NormalTok{)}
\FunctionTok{which.min}\NormalTok{(ave.scores)}
\end{Highlighting}
\end{Shaded}

\begin{verbatim}
hw3 
  3 
\end{verbatim}

\begin{Shaded}
\begin{Highlighting}[]
\NormalTok{tot.scores }\OtherTok{\textless{}{-}} \FunctionTok{apply}\NormalTok{(hw, }\DecValTok{2}\NormalTok{, sum, }\AttributeTok{na.rm=}\ConstantTok{TRUE}\NormalTok{)}
\FunctionTok{which.min}\NormalTok{(tot.scores)}
\end{Highlighting}
\end{Shaded}

\begin{verbatim}
hw2 
  2 
\end{verbatim}

\begin{Shaded}
\begin{Highlighting}[]
\NormalTok{tot.scores}
\end{Highlighting}
\end{Shaded}

\begin{verbatim}
 hw1  hw2  hw3  hw4  hw5 
1780 1456 1616 1703 1585 
\end{verbatim}

\begin{Shaded}
\begin{Highlighting}[]
\NormalTok{ave.scores}
\end{Highlighting}
\end{Shaded}

\begin{verbatim}
     hw1      hw2      hw3      hw4      hw5 
89.00000 80.88889 80.80000 89.63158 83.42105 
\end{verbatim}

\begin{quote}
It seems like hw3 was the toughest on students.
\end{quote}

\begin{quote}
Q4: Optional Extension: From your analysis of the gradebook, which
homework was most predictive of overall score (i.e.~highest correlation
with average grade score)? {[}1pt{]}
\end{quote}

\begin{Shaded}
\begin{Highlighting}[]
\NormalTok{hw}\SpecialCharTok{$}\NormalTok{hw1}
\end{Highlighting}
\end{Shaded}

\begin{verbatim}
 [1] 100  85  83  88  88  89  89  89  86  89  82 100  89  85  85  92  88  91  91
[20]  91
\end{verbatim}

\begin{Shaded}
\begin{Highlighting}[]
\NormalTok{ans}
\end{Highlighting}
\end{Shaded}

\begin{verbatim}
 student-1  student-2  student-3  student-4  student-5  student-6  student-7 
     91.75      82.50      84.25      84.25      88.25      89.00      94.00 
 student-8  student-9 student-10 student-11 student-12 student-13 student-14 
     93.75      87.75      79.00      86.00      91.75      92.25      87.75 
student-15 student-16 student-17 student-18 student-19 student-20 
     78.75      89.50      88.00      94.50      82.75      82.75 
\end{verbatim}

\begin{Shaded}
\begin{Highlighting}[]
\FunctionTok{cor}\NormalTok{(hw}\SpecialCharTok{$}\NormalTok{hw1, ans)}
\end{Highlighting}
\end{Shaded}

\begin{verbatim}
[1] 0.4250204
\end{verbatim}

\begin{Shaded}
\begin{Highlighting}[]
\FunctionTok{cor}\NormalTok{(hw}\SpecialCharTok{$}\NormalTok{hw3, ans)}
\end{Highlighting}
\end{Shaded}

\begin{verbatim}
[1] 0.3042561
\end{verbatim}

\begin{Shaded}
\begin{Highlighting}[]
\NormalTok{mask }\OtherTok{\textless{}{-}}\NormalTok{ hw}
\NormalTok{mask [}\FunctionTok{is.na}\NormalTok{(mask)] }\OtherTok{\textless{}{-}}\DecValTok{0}
\NormalTok{mask}
\end{Highlighting}
\end{Shaded}

\begin{verbatim}
           hw1 hw2 hw3 hw4 hw5
student-1  100  73 100  88  79
student-2   85  64  78  89  78
student-3   83  69  77 100  77
student-4   88   0  73 100  76
student-5   88 100  75  86  79
student-6   89  78 100  89  77
student-7   89 100  74  87 100
student-8   89 100  76  86 100
student-9   86 100  77  88  77
student-10  89  72  79   0  76
student-11  82  66  78  84 100
student-12 100  70  75  92 100
student-13  89 100  76 100  80
student-14  85 100  77  89  76
student-15  85  65  76  89   0
student-16  92 100  74  89  77
student-17  88  63 100  86  78
student-18  91   0 100  87 100
student-19  91  68  75  86  79
student-20  91  68  76  88  76
\end{verbatim}

\begin{Shaded}
\begin{Highlighting}[]
\FunctionTok{cor}\NormalTok{(mask}\SpecialCharTok{$}\NormalTok{hw5, ans)}
\end{Highlighting}
\end{Shaded}

\begin{verbatim}
[1] 0.6325982
\end{verbatim}

\begin{quote}
We can use the apply function here on the columns of hw and pass it the
overall scores for the class (in my `ans' object as an extra argument).
\end{quote}

\begin{Shaded}
\begin{Highlighting}[]
\FunctionTok{apply}\NormalTok{(mask, }\DecValTok{2}\NormalTok{, cor, }\AttributeTok{y=}\NormalTok{ans)}
\end{Highlighting}
\end{Shaded}

\begin{verbatim}
      hw1       hw2       hw3       hw4       hw5 
0.4250204 0.1767780 0.3042561 0.3810884 0.6325982 
\end{verbatim}

\begin{quote}
hw5 was most predictive of overall score because its correlation was
highest.
\end{quote}



\end{document}
